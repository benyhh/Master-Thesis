
\section{Experiment 1: Pointing Model using Neural Networks}

Table \ref{tab:exp1_rms_folds_best_model} show the RMS in arcseconds on all the folds for the different model arcitechtures.
The results are from the models with the smallest mean RMS for each of the arcitechtures.
From the mean RMS over all folds, we see that the different arcitechtures offer similar performance.
We also see that the RMS of fold $1$ is by far worse than the other folds.
The lowest mean RMS is from the architecture where the non-linear features are connected to the geometrical and harmonic features.\\

Table \ref{tab:exp1_hyperparameters_best_model} show the hyperparameter used for the best models.
All arcitechtures perform better with a single hidden layer.
The regular neural network uses ReLU activation and MSE loss, while the other arcitechtures use Tanh activation and MSD loss.
The regular neural network also have more neurons in the hidden layer and a higher learning rate.
The batch size is also varying. There seem to be some similarities between the hyperaprameters chosen for the three arcitectures with seperate features.
However, given the large standard deviation of the mean RMS, there is probably bigger issues than hyperparameter tuning.\\

Table \ref{tab:exp1_features} lists the features used in each of the best model for all the arcitechtures tested.

\begin{table}[!htbp]
    \centering
    \label{tab:exp1_rms_folds_best_model}
    \caption{RMS on all folds for the best model for all arcitechtures}
    \begin{tabular}{lcccccccc}
        \toprule
        & \multicolumn{6}{c}{RMS on test fold} & & \\
        \cmidrule(lr){2-7}
        Network & 1 & 2 & 3 & 4 & 5 & 6 & Mean & STD\\
        \midrule
        Regular & 28.06 & 19.34 & 12.28 & 13.25 & 17.19 & 16.33 & 17.74 & 5.19 \\
        Sep 1 & 30.69 & 16.93 & 13.76 & 10.04 & 15.77 & 13.61 & 16.80 & 6.57 \\
        Sep 2 & 27.34 & 20.75 & 12.65 & 24.17 & 13.64 & 16.69 & 19.21 & 5.38 \\
        Sep 3 & 30.27 & 20.59 & 14.01 & 10.76 & 13.67 & 10.34 & 16.61 & 6.97 \\
        \bottomrule
    \end{tabular}
\end{table}

\begin{table}[!htbp]
    \centering
    \label{tab:exp1_hyperparameters_best_model}
    \caption{Hyperparameters for the best model of each architecture}
    \begin{tabular}{lccccc}
        \toprule
        Architecture & Activation & Hidden Layers & Learning Rate & Batch Size & Loss \\
        \midrule
        Regular &  ReLU & [82] & 0.0199 & 334 & MSE \\
        Sep1    &  Tanh & [40] & 0.0098 & 101 & MSD \\
        Sep2    &  Tanh & [40] & 0.0098 & 101 & MSD \\
        Sep3    &  Tanh & [26] & 0.0039 & 358 & MSD \\
        \bottomrule
    \end{tabular}
\end{table}

\begin{table}[!htbp]
    \centering
    \caption{Features selected by hyperparameter search for each model}
    \label{tab:exp1_features}
    \begin{tabular}{|l|cccc|}
        \hline
        Feature & Sep 1 & Sep 2 & Sep 3 & Regular \\ \hline
        COMMANDAZ & x & x & x & x \\ \hline
        COMMANDEL & x & x & x & x \\ \hline
        DISP ABS3 MEDIAN 1 & x & x & x & x \\ \hline
        CA & x & x & x & \\ \hline
        NPAE & x & x & x & \\ \hline
        Constant & x & x & x & \\ \hline
        $\cos{(El)}$ & x & x & x & \\ \hline
        $\cos{(2\cdot El)}$ & x & x & x & \\ \hline
        $\cos{(3\cdot El)}$ & x & x & x & \\ \hline
        $\cos{(4\cdot El)}$ & x & x & x & \\ \hline
        $\cos{(5\cdot El)}$ & x & x & x & \\ \hline
        $\sin{(El)}$ & x & x & x & \\ \hline
        $\sin{(2 \cdot El)}$ & x & x & x & \\ \hline
        $\sin{(3 \cdot El)}$ & x & x & x & \\ \hline
        $\sin{(4 \cdot El)}$ & x & x & x & \\ \hline
        $\sin{(5 \cdot El)}$ & x & x & x & \\ \hline
        WINDSPEED VAR 5 & x & x & x & \\ \hline
        DEL TILTTEMP MEDIAN 1 & x & x & x & \\ \hline
        DAZ DISP MEDIAN 1 & & & x & \\ \hline
        POSITIONY MEDIAN 1 & & & x & \\ \hline
        TILT1Y MEDIAN 1 & & & x & \\ \hline
        TEMPERATURE MEDIAN 1 & & & x & \\ \hline
        TILT1T MEDIAN 1 & & & x & \\ \hline
    \end{tabular}
\end{table}

\section{Experiment 2: Pointing Correction Model}
Table \ref{tab:results_all_days} and \ref{tab:results_nflash_days} show the main results from experiment $2$.
They contain the average compared RMS \eqref{eq:mean_rms_compared} for azimuth and elevation models in case $1$ and $2$ for different numbers of features $k$ used to train the models.
We cleaned the datasets we used to train these models using the cleaning criteria and XGBoost classifier.
Table \ref{tab:results_all_days} shows results for models trained on scans from all instruments, while Table \ref{tab:results_nflash_days} shows results for models
trained on scans from NFLASH230 only. For both datasets, case $1$ does not offer any improvement at all.
On the other hand, models from case $2$ improve the pointing accuracy for most numbers of selected features $k$.

First, the results from models predicting offsets for all instruments.
Adding complexity seems to worsen the performance of azimuth models while improving the performance of elevation models.
The best performance for azimuth models is for $k=2$ with $\bar{r}_{RMS}=0.980$ and $\sigma_{\bar{r}} = 0.059$ which is a slight improvement. 
For elevation models, the best performance is for $k=50$ with $\bar{r}_{RMS}=0.955$ and $\sigma_{\bar{r}} = 0.029$.

Second, the results from models predicting offsets for NFLASH230 only.
The best models for azimuth are $k=2$ and $k=50$ with similar performance $\bar{r}_{RMS}=0.948$ and $\bar{r}_{RMS}=0.945$ respectively.
For elevation models, $k=50$ offers the best performance, with $\bar{r}_{RMS}=0.940$.
The standard deviations for these performance estimates are small.\\


We also conducted the same experiment for two other datasets, with the offsets and pointing corrections transformed to simulate a pointing correction after every scan.
We tested splitting the training and validation data completely randomly for all four datasets, unlike the presented results, where we dedicate a day to either training or validation.
For these results, see Appendix A (\ref{sec:appendix_a}). 

\begin{table}[!htbp]
    \centering
    \label{tab:results_all_days}
    \caption{$tmp2022\_clean\_clf\_results\_table$
    Resulting RMS from Case $1$ and $2$ for XGBoost model predicting pointing offset.
    The dataset used to get these results contain all scans and is cleaned using the regular criteria and the XGBoost classifier.
    The training and validation data is split on days, meaning that all the scans for a given day
    are in the training or validation set and not both. The test set is unaffected by this.}
    \begin{tabular}{ccccc c cccc}
        \toprule
        \multicolumn{1}{c}{} & \multicolumn{4}{c}{Case 1} & & \multicolumn{4}{c}{Case 2} \\
        \cmidrule(lr){2-5} \cmidrule(lr){7-10}
        \multicolumn{1}{c}{} & \multicolumn{2}{c}{Azimuth} & \multicolumn{2}{c}{Elevation} & & \multicolumn{2}{c}{Azimuth} & \multicolumn{2}{c}{Elevation} \\ 
        \cmidrule(lr){2-5} \cmidrule(lr){7-10}
        k & Mean & STD & Mean & STD & & Mean & STD & Mean & STD \\ 
        \midrule
         2 &     1.007 &     0.003 &     1.232 &     0.055 &  &  0.980 &     0.059 &     0.964 &     0.016 \\
         5 &     1.003 &     0.003 &     1.170 &     0.028 &  &  0.990 &     0.067 &     0.964 &     0.016 \\
        10 &     1.288 &     0.101 &     1.116 &     0.015 &  &  1.001 &     0.102 &     0.979 &     0.059 \\
        20 &     1.580 &     0.082 &     1.121 &     0.023 &  &  1.018 &     0.130 &     0.971 &     0.036 \\
        30 &     1.606 &     0.110 &     1.107 &     0.018 &  &  1.026 &     0.151 &     0.957 &     0.018 \\
        40 &     1.528 &     0.111 &     1.068 &     0.010 &  &  1.026 &     0.137 &     0.973 &     0.044 \\
        50 &     1.758 &     0.121 &     1.061 &     0.027 &  &  1.018 &     0.114 &     0.955 &     0.029 \\
        \bottomrule
    \end{tabular}
\end{table}

\begin{table}[!htbp]
    \centering
    \label{tab:results_nflash_days}
    \caption{$tmp2022\_clean\_clf\_nflash230\_results\_table$
    Resulting RMS from Case $1$ and $2$ for XGBoost model predicting pointing offset.
    The dataset used to get these results contain only NFLASH230 and is cleaned using the regular criteria and the XGBoost classifier.
    The training and validation data is split on days, meaning that all the scans for a given day
    are in the training or validation set and not both. The test set is unaffected by this.}
    \begin{tabular}{ccccc c cccc}
        \toprule
        \multicolumn{1}{c}{} & \multicolumn{4}{c}{Case 1} & & \multicolumn{4}{c}{Case 2} \\
        \cmidrule(lr){2-5} \cmidrule(lr){7-10}
        \multicolumn{1}{c}{} & \multicolumn{2}{c}{Azimuth} & \multicolumn{2}{c}{Elevation} & & \multicolumn{2}{c}{Azimuth} & \multicolumn{2}{c}{Elevation} \\ 
        \cmidrule(lr){2-5} \cmidrule(lr){7-10}
        k & Mean & STD & Mean & STD & & Mean & STD & Mean & STD \\ 
        \midrule
         2 &     0.982 &     0.014 &     1.020 &     0.024 &  &  0.948 &     0.056 &     0.972 &     0.081 \\
         5 &     1.366 &     0.077 &     1.198 &     0.034 &  &  0.983 &     0.142 &     0.953 &     0.097 \\
        10 &     1.383 &     0.087 &     1.155 &     0.047 &  &  0.957 &     0.080 &     0.967 &     0.087 \\
        20 &     1.252 &     0.119 &     1.126 &     0.071 &  &  0.972 &     0.131 &     0.949 &     0.069 \\
        30 &     1.335 &     0.226 &     1.094 &     0.041 &  &  0.963 &     0.093 &     0.959 &     0.077 \\
        40 &     1.146 &     0.036 &     1.058 &     0.020 &  &  0.961 &     0.089 &     0.948 &     0.077 \\
        50 &     1.202 &     0.131 &     1.062 &     0.022 &  &  0.945 &     0.073 &     0.940 &     0.075 \\
        \bottomrule
    \end{tabular}
\end{table}



Table \ref{tab:results_minval_days04} show the performane of the pointing correction model if we choose always choose the model with the lowest validation
loss on and use that model for the test set. It should the performance estimate and standard deviation for the two different cases, and for different datasets.
The data training and validation data is split on whole days, and the validation size is $0.43\%$ of the training/validation set.
This does not offer any improvement over the result presented above, but for an NFLASH230 model, this still proves to be a viable strategy.
The performance using this strategy for NFLASH230 is $\bar{r}_{RMS}=0.958$ for azimuth and $\bar{r}_{RMS}=0.941$ for elevation. 
Standard deviations are also small.

For a list of the $50$ features with the most mutual information to the target variable, see \ref{tab:exp2_top50_features} in the Appendix \ref{sec:appendix_a}.



\begin{table}[!htbp]
    \centering
    \label{tab:results_minval_days04}
    \caption{Performance when choosing min validation for each fold. Train/val split on days. Test size $0.43$.}
    \begin{tabular}{lcccc c cccc}
        \toprule
        \multicolumn{1}{c}{} & \multicolumn{4}{c}{Case 1} & & \multicolumn{4}{c}{Case 2} \\
        \cmidrule(lr){2-5} \cmidrule(lr){7-10}
        \multicolumn{1}{c}{} & \multicolumn{2}{c}{Azimuth} & \multicolumn{2}{c}{Elevation} & & \multicolumn{2}{c}{Azimuth} & \multicolumn{2}{c}{Elevation} \\ 
        \cmidrule(lr){2-5} \cmidrule(lr){7-10}
        Dataset &  Mean &  STD &  Mean &  STD & & Mean &  STD &  Mean &  STD \\
        \midrule
        Transformed       &     2.140 &     0.498 &     1.073 &     0.040 &  &   1.008 &     0.073 &     0.949 &     0.022 \\
        Transformed NF230 &     1.242 &     0.047 &     1.006 &     0.009 &  &   0.999 &     0.096 &     1.074 &     0.201 \\
        Regular           &     1.730 &     0.126 &     1.170 &     0.028 &  &   1.016 &     0.114 &     0.994 &     0.065 \\
        Regular N230      &     1.251 &     0.131 &     1.198 &     0.033 &  &   0.958 &     0.055 &     0.941 &     0.079 \\
        \bottomrule
    \end{tabular}
\end{table}

