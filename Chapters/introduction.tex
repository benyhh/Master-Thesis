Radio/(sub)-millimeter telescopes are powerful tools to study the universe at radio, sub-millimeter, and millimeter wavelengths.
These telescopes are designed to capture and detect electromagnetic radiation from space, which can provide valuable insights into a range of astronomical phenomena,
from the formation of stars and galaxies to the behavior of black holes.
One of the key components of a radio telescope is its reflective surface, which collects and focuses the incoming radiation.
Most radio/(sub)-mm telescopes have a large, parabolic dish-shaped primary mirror, reflecting incoming radiation onto a smaller, secondary mirror.
The secondary mirror then reflects the radiation onto a detector or receiver placed at the focal point, which records and processes the signals.
Most radio/(sub)-mm instruments do not have an imaging camera.
Hence, the correct positioning of the source within the resolution element (beam) and at the center of the field of view cannot be checked directly.
So-called "pointing" measurements are required to fit the source at the center of the beam.
Radio/(sub)-mm telescopes detect and record photons over time, which are then processed to create a composite image or spectrum.
However, this process requires highly accurate pointing, as even slight errors in the telescope's orientation can significantly affect the resulting data quality.
Pointing errors, often referred to as pointing offsets, can be caused by various factors, including thermal deformation of the telescope components,
gravitational deformation, and other environmental factors like humidity and wind. As these factors may change over time, the offsets are also affected.
To achieve this accuracy, radio/(sub)-mm telescopes use pointing models, which take into account a range of factors that can contribute to the pointing error,
including weather conditions, telescope structure, and the target's position in the sky.
The APEX telescope, located in the high-altitude Atacama Desert in Chile,
currently uses an effective analytical pointing model that still requires regular corrections based on recent observations of pointing offsets.
This research aims to investigate the use of observational data, such as weather patterns and telescope pointing,
to create a more comprehensive pointing model that factors in the influence of these variables on pointing accuracy.
Furthermore, the research will explore using machine learning models to replace the analytical model at APEX.
A machine learning approach would benefit larger radio/(sub)-mm telescopes like the future AtLAST telescope
(which will have a $50$-meter diameter primary mirror and $12$-meter diameter subreflector, \hyperlink{https://www.atlast.uio.no}{link to website}),
where an analytical model will not be available due to the complexity of the telescope's systems.
By developing a more advanced and reliable pointing model, this research will enhance the capabilities of current and future radio/(sub)-mm telescopes to advance our understanding of the universe.
\textcolor{red}{Add in chapter overview when the chapters are fixed}
