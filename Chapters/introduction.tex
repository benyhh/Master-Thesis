Radio/(sub)-millimeter telescopes are powerful tools used to study the universe at radio, sub millimeter and millimeter wavelengths.
These telescopes are designed to capture and detect electromagnetic radiation from space, which can provide valuable insights into a range of astronomical phenomena,
from the formation of stars and galaxies to the behavior of black holes.
One of the key components of a radio telescope is its reflective surface, which collects and focuses the incoming radiation.
Most radio/(sub)-mm telescopes have a large, parabolic dish-shaped primary mirror, which reflects the incoming radiation onto a smaller, secondary mirror.
The secondary mirror then reflects the radiation onto a detector or receiver, which records and processes the signals.
Unlike optical telescopes, which provide real-time images of what they observe, radio/(sub)-mm telescopes detect and record photons over time,
which are then processed to create a composite image or spectrum.
However, this process requires highly accurate pointing, as even slight errors in the telescope's orientation can significantly affect the quality of the resulting data.
Pointing errors, often referred to as pointing offsets, can be caused by a variety of factors, including thermal deformation of the telescopes components,
gravitational deformation, and other environmental factors like humidity and wind. As these factors may change over time, the offsets naturaly change as well.
To achieve this accuracy, radio/(sub)-mm telescopes use pointing models, which take into account a range of factors that can affect pointing accuracy,
including weather conditions, telescope structure, and the position of the target in the sky.
The APEX telescope, located in the high-altitude Atacama Desert in Chile,
currently uses an analytical pointing model that is effective but still requires regular corrections based on recent observations of pointing offsets.
This research aims to investigate the use of sensory data, such as weather data and tiltmeters,
to create a more comprehensive pointing model that accounts for the effects of these variables on pointing accuracy.
Furthermore, the research will explore the use of machine learning models to replace the analytical model at APEX in its entirety.
This would be particularly useful for larger radio/(sub)-mm telescopes like the future AtLAST telescope, where an analytical model will not be available due to the complexity of the telescope's systems.
By developing a more advanced and reliable pointing model, this research will contribute to enhancing the capabilities of current and future radio/(sub)-mm telescopes, used to advance our understanding of the universe.
\textcolor{red}{Add in chapter overview when the chapters are fixed}