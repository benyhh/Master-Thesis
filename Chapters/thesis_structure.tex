The thesis begins by providing an overview of some of the related work in the field. Next, we introduce the necessary astronomical background.
This chapter also includes a detailed description of the APEX telescope database and how the current pointing model was developed.
The following chapter focuses on the machine learning concepts that were employed in the thesis. We provide a comprehensive explanation of the theories used.
We then move on to describe the methods we used to pursue the two research questions.
The subsequent sections of the thesis present the results of our research, followed by a detailed discussion and conclusion.

The first chapter of the thesis introduces the problem of pointing with radio telesopes, along with the research questions, related works, and some astronomy background.
The rest of the thesis is then devided into three parts. Part I, Background \& Theory, has two chapters. The first describing the pointing model developed at APEX and the data we used for the research,
and the second presenting the machine learning concepts we used for the research.
Part II, Methods & Analysis, had three chapters. The first chapter describes the data processing and feature engineering. The second chapter  
The thesis is then devided into three parts. The first part is background