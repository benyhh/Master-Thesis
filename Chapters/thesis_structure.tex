The thesis is structured into three parts, with the first chapter introducing the problem of pointing with radio telescopes.
The first chapter also discusses the research questions, related works, and astronomy background.

Part I, Background \& Theory, consists of two chapters.
Chapter $2$ provides a detailed description of the pointing model developed at APEX and the data we used for the research in this thesis.
Chapter $3$ presents the machine learning concepts we utilized.

Part II, Data Processing \& Methods, also contains two chapters.
Chapter $4$ outlines the data processing and feature engineering techniques we used in the study.
Chapter $5$ describes two machine learning experiments that we aimed at answering the research questions posed in the introduction.

Finally, Part III, Results \& Discussion, includes chapter $6$ with the results and discussion of the two experiments, followed by the conclusion in chapter $7$.
In addition to the main body of the thesis, we provide an appendix, which includes other results from the research.