The use of machine learning in astronomy has gaiend popularoty in recent years for a variety of purposes, such as data analysis and prediction.
However, its application in the context of pointing models in radio telescopes has not yet been explored extensively. 
In this section, we provide a review of existing literature on pointing models in radio telescopes, as well as the use of machine learning for similar applications

\section{Pointing Models in Radio Telescopes}
Traditional methods for pointing models in radio telescopes involve modeling the pointing error as a function of various parameters, 
such as azimuth, elevation, temperature, and time. 
These models are often complex and require significant effort to develop and mantain.
Moreover, they can be limited by the limited by the accuracy of the models used atmospheric refraction, instrumental error, and other sources of noise.

Several papers have described various approaches to improve the pointing accuracy of radio telescopes.
For example, \cite{white_green_2022} constructs a pointing model for the Green Bank Telescope
using theoretical terms based on the telescope's structure and analysis on the thermal deformation of the telescope structure.
\cite{greve1996pointing} also studies seasonal effects on the pointing.



\section{Machine Learning for Pointing Models}

\section{Challenges and Opportunities}




\section{Pointing Models in Radio Telescopes}
Traditional methods for pointing models in radio telescopes involve modeling the pointing error as a function of various parameters,
such as azimuth, elevation, temperature, and time.
For example, \cite{white_green_2022} constructs a pointing model for the Green Bank Telescope
using theoretical terms based on the telescope's structure and analysis of data from the metrology systems. 

The paper \cite{greve1996pointing} also studies seasonal effects on the pointing.
A problem with these models is that they require significant effort to develop and mantain.

\section{Machine Learning for Pointing Models}

\section{Challenges and Opportunities}

