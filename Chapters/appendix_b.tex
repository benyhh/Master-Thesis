\section{Transformation of Pointing Offsets and Corrections}

Table \ref{tab:offset_and_corrections} show examples of pointing offsets and corrections applied during the pointing scans.
The column "Original" show raw pointing scan data.
The pointing corrections $ca$ and $ie$ are normally updated according to equations \eqref{eq:ca} and \eqref{eq:ie}, 
as shown in the first two rows of the table.
However, observers may choose not to update the pointing, particularly when the pointing offset is small, as illustrated in the consecutive row of the table.
In other cases, the corrections may be updated but not according to equations \eqref{eq:ca} and \eqref{eq:ie}.
This can occur when a new science project is loaded and the pointing correction from the previous time that project was used is applied.
These factors introduce several challenges for the training of a model:
\begin{itemize}
    \item $ca$ and $ie$ should represent the optimal correction using all the information we have about the current state of a system.
    If we do not update the corrections, there is some information about the system (the previously observed pointing offset) that the model is not receiving.
    \item Some features are constructed as the change in variables since the last correction.
    If the corrections are not updated, this interval is longer than if they were, and the resulting features could be more prone to uncertainties and noise.
    A problem with the integration also occurs if the corrections are not updated according to the equations \eqref{eq:ca} and \eqref{eq:ie}.
    Then, we do not know when those corrections represent the system, resulting in inaccurate features.
\end{itemize}

A possible solution to this problem is a two step procedure where we first transform the offsets and corrections to represent the system at the most recent pointing scan,
and second use these transformed offsets as training labels and the transformed corrections as training inputs.

In practice, we do this by assuming the corrections $ca$ and $ie$ are updated after every pointing scan.
This changes the correction applied during the next pointing scan, which further affects the observed pointing offset.
This effect propagates throughout the whole dataset.
The following formulas 
\begin{align}
    \Tilde{ca}_i &= \Tilde{ca}_{i-1} + \Tilde{\delta}_{az,i-1}\label{eq:ca_tilde}\\
    \Tilde{ie}_i &= \Tilde{ie}_{i-1} - \Tilde{\delta}_{el,i-1\label{eq:ie_tilde}}\\
    \Tilde{\delta}_{az,i} &= \delta_{az,i} + ca_i - \Tilde{ca}_i\label{eq:off_az_tilde}\\
    \Tilde{\delta}_{el,i} &= \delta_{el,i} - ie_i + \Tilde{ie}_i\label{eq:off_el_tilde}
\end{align}
Where the "$\sim$" denotes a transformed variable.
Using these transformations,
the corrections used for training and the resulting offset are similar to the ones that would be observed if the corrections were made according to the equations
\eqref{eq:ca} and \eqref{eq:ie} after every pointing scan. The column "Transformed" in the table shows the transformed variables.

\begin{table}[H]
    \centering
    \caption{\textbf{Original:} Example from the dataset of the observed pointing offsets and the corrections applied during the pointing scan.
    \textbf{Transformed:} Pointing offsets and corrections according to equations \eqref{eq:ca_tilde}, \eqref{eq:ie_tilde}, \eqref{eq:off_az_tilde}, and \eqref{eq:off_el_tilde}.}
    \label{tab:offset_and_corrections}
    \begin{tabular}{c cccc c cccc}
    \toprule
    \multicolumn{1}{c}{} & \multicolumn{4}{c}{Original} & \multicolumn{1}{c}{} & \multicolumn{4}{c}{Transformed} \\
    \cmidrule(lr){2-5} \cmidrule(lr){7-10} 
    $i$ &  $\delta_{az}$ &  $\delta_{el}$ &  $ca$ & $ie$  & & $\Tilde{\delta}_{az}$ &  $\Tilde{\delta}_{el}$ &  $\Tilde{ca}$ &  $\Tilde{ie}$ \\
    \midrule
    $1$ &     $1.2$ & $0.1$ & $2.1$ &  $1.7$ &   &    $1.2$ &       $0.1$ &       $2.1$ &       $1.7$ \\
    $2$ &     $0.0$ & $0.5$ & $3.3$ &  $1.6$ &   &    $0.0$ &       $0.5$ &       $3.3$ &       $1.6$ \\
    $3$ &    $-1.1$ & $0.0$ & $3.3$ &  $1.6$ &   &   $-1.1$ &      $-0.5$ &       $3.3$ &       $1.1$ \\
    $4$ &     $0.6$ & $0.7$ & $2.2$ &  $1.6$ &   &    $0.7$ &       $0.7$ &       $2.2$ &       $1.6$ \\
    $5$ &     $0.9$ & $1.4$ & $2.2$ &  $1.6$ &   &    $0.2$ &       $0.7$ &       $2.8$ &       $0.9$ \\
    $6$ &     $1.0$ & $1.1$ & $2.2$ &  $1.6$ &   &    $0.1$ &      $-0.3$ &       $3.1$ &       $0.2$ \\
    $7$ &    $-0.9$ & $1.2$ & $3.1$ &  $0.5$ &   &   $-1.0$ &       $1.3$ &       $3.2$ &       $0.5$ \\
    $8$ &     $0.5$ & $1.5$ & $2.2$ & $-0.7$ &   &    $0.5$ &       $1.4$ &       $2.2$ &      $-0.7$ \\
    $9$ &    $-0.3$ & $0.4$ & $2.2$ & $-0.7$ &   &   $-0.8$ &      $-1.1$ &       $2.7$ &      $-2.2$ \\
    \bottomrule
    \end{tabular}
\end{table}